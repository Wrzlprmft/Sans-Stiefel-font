\documentclass[a4paper, 17pt, onecolumn]{extarticle}

\usepackage{xltxtra}

\usepackage{array, calc, color, enumitem, graphicx, graphics, hyphenat, icomma, ngerman, relsize, tabularx, tikz, xifthen, xspace}

\setromanfont[Mapping=tex-text]{Sans Stiefel}
\setsansfont[Mapping=tex-text, Scale=MatchLowercase]{Linux Biolinum O}
\setmonofont[Scale=MatchLowercase]{DejaVu Sans Mono}

\setlogokern{eL}{0.0em} 
\setlogokern{aT}{-0.075em}

\usepackage{polyglossia}
\setdefaultlanguage[spelling=new, latesthyphen=true, script=fraktur, babelshorthands=true]{german}

\usepackage[colorlinks=true, linkcolor=black, citecolor=black, urlcolor=black]{hyperref}


% ----Seite einstellen----
\setlength{\textwidth}{17 cm}
\setlength{\hoffset}{-1 in}
\setlength{\oddsidemargin}{2 cm}

\setlength{\voffset}{-1 in}
\setlength{\topmargin}{1.6 cm}
\setlength{\headsep}{0 pt}
\setlength{\headheight}{0 pt}
\setlength{\textheight}{26.5 cm}

\pagestyle{plain}


% ----Textsatzregeln----
\hyphenpenalty=500
\pretolerance=150
\tolerance=1500
\setlength{\emergencystretch}{\textwidth}
\frenchspacing
\renewcommand{\/}{\ZWNJhyp{}}
\renewcommand{\bfseries}{\xspace\addfontfeature{LetterSpace=15.0,WordSpace=1.5,Ligatures={Required,NoCommon}}}


% ----Für Tabellen----
\newcolumntype{C}{>{\centering\arraybackslash}X}
\renewcommand{\arraystretch}{1.3}


% ----Hilfsbefehle----
\newlength{\lw}
\newenvironment{Kasten}[2][2pt]{%
	\begin{center}
		\setlength{\lw}{#1}%
		\begin{tikzpicture}[line width=\lw, line join=round]%
			\node [inner sep=3\lw, text width=#2-10\lw]%
				(BOXCONTENT)%
				\bgroup\ignorespaces%
}{
				\egroup;
			\draw ([xshift=-\lw]BOXCONTENT.north west) -- ([xshift=\lw]BOXCONTENT.north east);
			\draw ([xshift=-\lw]BOXCONTENT.south west) -- ([xshift=\lw]BOXCONTENT.south east);
			\draw ([xshift=-\lw, yshift=-2\lw]BOXCONTENT.north west) -- ++(\lw,0) -- ([yshift=2\lw]BOXCONTENT.south west) -- ++(-\lw,0);
			\draw ([xshift=\lw, yshift=-2\lw]BOXCONTENT.north east) -- ++(-\lw,0) -- ([yshift=2\lw]BOXCONTENT.south east) -- ++(+\lw,0);
		\end{tikzpicture}
	\end{center}
}

\newcommand{\Beispiel}[2][]{%
\normalsize%
\ifthenelse{\isempty{#1}}{\verb⚥#2⚥}{\ifthenelse{\isin{=}{#1}}{\verb⚥#1⚥}{#1}}%
\strut\linebreak\LARGE\strut%
\ifthenelse{\isin{=}{#1}}{\addfontfeature{#1}}{}#2%
}


% ----Metadata----
\title{Sans Stiefel}
\author{Gerrit Ansmann}
\date{\today}

% ------------------------------------------

\begin{document}

\hyphenation{Schaft-ſtie-fel-gro-tesk Schaft-ſtie-fel-gro-tes-ken}

\begin{Kasten}[4pt]{0.61\linewidth}
	\centering
	\textlarger[4]{\textbf{Sans }\addfontfeature{CharacterVariant=1:1} \textbf{ Stiefel}}\\[\baselineskip]
	\textlarger[1]{OpenType-Funktionen\linebreak mit Beiſpielen für {\addfontfeature{StylisticSet=1} \XeLaTeX}}\\[\baselineskip]	
	Gerrit Ansmann, \today

\end{Kasten}

\vfill

\noindent Dieſe Schrift iſt ein Verſuch, die Frage zu beantworten, wie ſich die gebrochenen Schriften hätten weiterentwickeln können, wären ſie nicht 1941 abgeſchafft worden.
Sie befindet ſich irgendwo zwiſchen einer Fraktur und einer Grotesken (genauer geſagt einer Egyptienne) und ordnet ſich damit in die gebrochenen Grotesken ein.
Dieſe auch »Schaftſtiefelgrotesken« genannten Schriften werden ſtärker noch als die klaſſiſchen Frakturen mit dem Nationalſozialismus aſſoziiert und beinhalten u.\,a. Vertreter namens »Großdeutſch« und »National«.

Ich habe die vorliegende Schrift Sans Stiefel (»ohne Stiefel«) genannt, da trotz des vergleichbaren Anſatzes deutliche optiſche Unterſchiede zu beſagten Schaftſtiefelgrotesken beſtehen — ſei letztere Bezeichnung nun angebracht oder nicht.
Wer darauf beſteht, Schriften, Schuhwerk und Ideologien miteinander zu aſſoziieren, darf den Namen gerne auch als Distanzierung von mit Stiefeln aſſoziierten Ideologien auf\/faſſen.



\section{Enthaltene Zeichen}
Sans Stiefel enthält den Unicode-Bereich \textsf{Basic Latin} vollſtändig ſowie bis auf wenige Ausnahmen die Bereiche \textsf{Latin"~1 Supplement}, \textsf{Latin Extended"~A} und \textsf{Number Forms}.
Außerdem ſind ungefähr die Hälfte der Zeichen aus \textsf{General Punctuation}, hoch- und tiefgeſtellte Ziffern ſowie einige weitere nicht allzu ſelten auf\/tauchende Zeichen wie €, №, ℗, ℠, ™, ℓ, μ, π, ꝛ, →, ⇒, ∓, ∝, ∞, ≈, ∼, ≡, ≫, ≥ oder ⊥ enthalten.

\section{OpenType-Funktionen}
Wenn bei den Beiſpielen auf den folgenden Seiten der Quelltext nicht offenſichtlich mit dem Beiſpiel korreſpondiert, ſo iſt er als \textsf{fontspec}-Einſtellung zu verſtehen und kann z.\,B. als Argument an \texttt{\textbackslash addfontfeature} übergeben werden.

\subsection{Ligaturen und Kontextabhängiges}
In Sans Stiefel ſind zur Zeit 55~Ligaturen ſtandardmäßig aktiviert. Hier\/zu zählen:
\begin{itemize}[itemsep = 0pt, parsep=\fill]
	\item im Deutſchen häufig verwendete Ligaturen wie ch, ck, Ch, ff, fi, ft, kt, rz, ſi, ſſ, ſt, tt, tz oder Th,
	\item ſeltene wie ct, fj, kk, rꝛ, ſk oder ſl,
	\item vermutlich nur in Namen oder Kunſtwörtern auf\/tauchende wie fh, fk, ftz oder Tl,
	\item für Fremdſprachen wie (Alt"~)Engliſch, Isländiſch oder Niederländiſch gedachte wie ſh, fþ, fð oder ſij,
	\item für andere Paradigmata der ſ"~Schreibung (einmal mehr: Engliſch) gedachte wie ſſb,
	\item ſolche, bei denen ſelbſt ich bisher noch keine Idee habe, wozu ſie gut ſein könnten, wie ſtz.
\end{itemize}
All dieſe Ligaturen ſind mit dem OpenType-Feature \texttt{liga} verknüpft und können mit der \textsf{fontspec}-Einſtellung \texttt{Ligatures=NoCommon} deaktiviert werden, wovon aber abzuraten iſt:
\begin{Kasten}{0.9\linewidth}
	\begin{tabularx}{\linewidth}{C C}
		\Beispiel[Standard]{offenſichtlich Schmerzmittel Charakterſtück}& \Beispiel[Ligatures=NoCommon]{offenſichtlich Schmerzmittel Charakterſtück}
	\end{tabularx}
\end{Kasten}

Einzelne Ligaturen können mit einem Bindehemmer (im Beiſpiel \verb⚥\/⚥) unterbunden werden:
\begin{Kasten}{0.75\linewidth}
	\begin{tabularx}{\linewidth}{C C}
		\Beispiel{Ur\/zeit\/tier} & \Beispiel{Urzeittier}\\
		\Beispiel{Auf\/lauf\/kind} & \Beispiel{Auflaufkind}
	\end{tabularx}
\end{Kasten}

Die Ligaturen ch, ck, ſt und tz ſind zuſätzlich mit dem OpenType-Feature \texttt{rlig} verknüpft, um automatiſches Sperren zu ermöglichen, z.\,B. via:
\begin{verbatim}
\renewcommand {\bfseries} {\xspace \addfontfeature{
       LetterSpace = 15.0, WordSpace = 1.5,
       Ligatures = {Required, NoCommon} }} 
\end{verbatim}

Ähnlich den Ligaturen gibt es ſchließlich noch eine kurze Form des~ſ für den Fall, daſs dieſem ein beinahe kollidierender Buchſtabe wie ä folgt und eine Ligatur oder poſitives Kerning keine Option ſind:
\begin{Kasten}{0.55\linewidth}
	\begin{tabularx}{\linewidth}{C C}
		\Beispiel{ſüffiſant} & \Beispiel{ſ\/üffiſant}
	\end{tabularx}
\end{Kasten}

\subsection{alternatives S}
Mit dem OpenType-Feature \texttt{cv01} iſt eine alternative Form des S verknüpft:
\begin{Kasten}{0.75\linewidth}
	\begin{tabularx}{\linewidth}{>{\hsize=.7\hsize}C >{\hsize=1.3\hsize}C}
		\Beispiel[Standard]{SŚŜŠŞ} & \Beispiel[CharacterVariant=1:1]{SŚŜŠŞ}
	\end{tabularx}
\end{Kasten}

\subsection{Antiqua-Majuskeln}
In den meiſten gebrochenen Schriften ſind Großbuchſtaben unſchön und ſchwer zu leſen, wenn ſie im Verbund auf\/treten. Stattdeſſen wird normalerweiſe eine externe Antiqua genutzt. Für die in der deutſchen Sprache auf\/tretenden Großbuchſtaben (A–Z, Ä, Ö, Ü, ẞ) enthält Sans Stiefel jedoch eigene Antiqua-Verſionen. Dieſe werden mittels OpenType automatiſch genutzt, ſobald zwei Großbuchſtaben aufeinanderfolgen, können aber auch über das OpenType-Feature \texttt{ss01} gezielt aktiviert werden:
\begin{Kasten}{0.9\linewidth}
	\begin{tabularx}{\linewidth}{>{\hsize=.75\hsize}C >{\hsize=1.25\hsize}C}
		\Beispiel{MAJUSKEL} &\Beispiel{M\/A\/J\/U\/S\/K\/E\/L} \\
		\Beispiel[Standard]{XeLaTeX} & \Beispiel[StylisticSet=1]{XeLaTeX}
	\end{tabularx}
\end{Kasten}
 
\subsection{Ziffern}
Über die OpenType-Features \texttt{onum} und \texttt{tnum} können Minuskel- bzw. Tabellenziffern angeſteuert werden, wobei ſich im letzten Fall nur die Eins ändert.
\begin{Kasten}{\linewidth}
	\begin{tabularx}{\linewidth}{lCC}
		\verb⚥Numbers=⚥&
		\verb⚥Proportional⚥ &
		\verb⚥Monospaced⚥ \\[0.3\baselineskip]
		\verb⚥Lining⚥ &
		\huge 0123456789 &
		{\huge \addfontfeature{Numbers=Monospaced} 0123456789} \\[0.3\baselineskip]
		\verb⚥OldStyle⚥ &
		{\huge \addfontfeature{Numbers=OldStyle} 0123456789} &
		{\huge \addfontfeature{Numbers={Monospaced, OldStyle}} 0123456789}
	\end{tabularx}
\end{Kasten}

\end{document}
